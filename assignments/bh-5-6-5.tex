\subsection{Example BH.5.6.5}
Read example  BH.5.6.5 first.
We chop up the exercise in many small steps.


For the code below, run it first for a small number of samples (Here I choose \texttt{samples=2}). Then read (and if necessary put in more) print statements to see what results you get.  Use the output to answer the questions below.

\begin{minted}[]{python}
import numpy as np
from scipy.stats import expon

np.random.seed(10)

labda = 6
num = 3
samples = 2

X = expon(scale=labda).rvs((samples, num))
print(X) #this
T = np.sort(X, axis=1)
print(T)
print(T.mean(axis=0))

expected = np.array([labda / (num - j) for j in range(num)])
print(expected)
print(expected.cumsum())
\end{minted}


\begin{minted}[]{R}
set.seed(10)

labda = 6
num = 3
samples = 2

X = matrix(rexp(samples * num, rate = 1 / labda), nrow = samples, ncol = num)
print(X) #this
bigT = X
for (i in 1:samples) {
  bigT[i,] = sort(bigT[i,])
}
print(bigT)
print(colMeans(bigT))

expected = rep(0, num)
for (j in 1:num) {
  expected[j] = labda / (num - (j - 1))
}
print(expected)
print(cumsum(expected))
\end{minted}



\begin{exercise}
In line P.11 (R.8),\footnote{Line P.x refers to line x of the Python code.
  Line R.x refers to line x of the R code.}
also marked as 'this', we print the value of \texttt{X} in line P.10 (R.7), respectively.
What is the meaning of \texttt{X}?
\end{exercise}

\begin{exercise}
What is the meaning of \texttt{T} in line P.12 (R.11)?
\end{exercise}


\begin{exercise}
What do we print in line P.14 (R.14)?
\end{exercise}

\begin{exercise}
What is meaning of the variable \texttt{expected}?
\end{exercise}

\begin{exercise}
 What is the \texttt{cumsum} of \texttt{expected}?
\end{exercise}

\begin{exercise}
 Now that you understand what is going on, rerun the simulation for a larger number of samples, e.g., 1000, and discuss the results briefly.
\end{exercise}
