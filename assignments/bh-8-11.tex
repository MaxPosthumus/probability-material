\input{header}

\subsection{BH.8.11}
\label{sec:bh-8.1}

We apply the results of BH.8.11 to various examples to check (numerically) whether the result is indeed ok.

\begin{minted}[]{python}
import numpy as np
import matplotlib.pyplot as plt


def ecdf(x):
    support, values = np.unique(x, return_counts=True)
    X = values.cumsum()
    return support, X / X[-1]


rng = np.random.default_rng(2021)

# experiment 1

num = 1000
labda = 0.5
X = rng.exponential(1 / labda, size=num)
Y = 1 / X
s, y = ecdf(Y)
plt.xlim(0, 10)
plt.plot(s, y, label="ecdf")
plt.plot(s, np.exp(-labda / s), label="theory")
plt.legend()
plt.savefig("inv_exp.pdf")
\end{minted}

\begin{exercise}
In the code above, what is the distribution of $X$?
\end{exercise}

\begin{exercise}
Which of our variables correspond to $T$ and $V$ of BH.8.11?
\end{exercise}

\begin{exercise}
What is the form for $F_{V}$ when you appy the result of BH.8.11 to the case of this exercise? Take all parts of the support of the relevant rvs.
\end{exercise}

The next code is for the next few questions.

\begin{minted}[]{python}
T = rng.exponential(1 / labda, size=num) - 3
V = 1 / T
s, y = ecdf(V)


def FT(x):
    if x < -3:
        return 0
    return 1 - np.exp(-labda * (x + 3))  # why not -3?


def FV(v):
    if v > 0:
        return FT(0) + 1 - FT(1 / v)
    if v == 0:
        return FT(0)
    return FT(0) - FT(1 / v)


yy = [FV(v) for v in s]

plt.clf()
plt.xlim(-10, 10)
plt.plot(s, y, label="ecdf")
plt.plot(s, yy, label="theory")
plt.legend()
plt.savefig("inv_exp3.pdf")
\end{minted}

\begin{minted}[]{R}
\end{minted}

\begin{exercise}
In the code we marked a line with the comment \texttt{why not -3}? So, why do we have $+3$ instead of $-3$?
\end{exercise}

\begin{exercise}
Comment on the figures you get.
\end{exercise}

\begin{exercise}
Remove the line \texttt{xlim}. What happens to the figure? Why do we include these limits?
\end{exercise}



\begin{exercise}
Use the above code to plot $1/T$ when $T\sim \Norm{3, 1}$ on the interval $[-1, 1]$.
Why do you have to change the code for \texttt{FT}?
Include your code for \texttt{FT}. (Hint: use \texttt{from scipy.stats import norm} and then the cdf. Search the web for example to see what you have to type.)
Include the figure in your report.
\end{exercise}





\input{trailer}