% arara: pdflatex: { shell: yes }
% arara: pythontex: {verbose: yes, rerun: modified }
% arara: pdflatex: { shell: yes }
% arara: move: {  files: ['assignments.pdf'], target: ['../pdf-files'] }


\documentclass[assignments]{subfiles}

% \documentclass[a4paper,12pt]{article}


% %\usepackage[nosolutions]{optional}
% %\usepackage[justhints]{optional}
% %\usepackage[check]{optional}
% \usepackage[all-solutions-at-end]{optional}

% \usepackage{../common/preamble}
% %\newcommand{\sectionbreak}{\clearpage}
% \setcounter{tocdepth}{2}
% \usepackage{a4wide}
% \usepackage{../common/abbreviations}


% \newtheorem{theorem}{Theorem}[section]
% \newtheorem{exercise}[theorem]{Ex}
% \newtheorem{remark}[theorem]{Remark}


% % \Opensolutionfile{hint}
% % \Opensolutionfile{ans}



% \author{PD team}
% \date{\today}
% \title{Probability distributions EBP038A05\\
% Assignments}

\begin{document}
\maketitle
\tableofcontents

\subfile{intro.tex}


\section{Assignment 1}

Mind:
\begin{itemize}
\item Line P.x refers to line x of the Python code.   Line R.x refers to line x of the R code.
\item  When two pieces of code are given and one is in parentheses, the first one is the Python code and the one in parentheses is the R code. So if you use Python you have to explain what  \texttt{A[1:]} does; if you use R you have to explain what \texttt{A[-1]} does.
\end{itemize}


%\subfile{bh-5-exp.tex}
\subfile{bh-5-6-5.tex}
\subfile{bh-7-1.tex}
%\subfile{bh-7-9.tex}


\section{Assignment 2}
\subfile{bh-7-48.tex}
%\subfile{bh-7-53.tex}
\subfile{bh-7-86.tex}


\section{Assignment 3}
%\subfile{bh-8-4-figure.tex}
%\subfile{bh-8-4-3.tex}
\subfile{bh-8-1.tex}
%\subfile{bh-8-18.tex}



\section{Assignment 4}
%\subfile{bh-8-27.tex}
%\subfile{bh-8-36.tex}
%\subfile{bh-8-54.tex}
\subfile{bh-9-1.tex}


\section{Assignment 5}
\subfile{bh-9-1-7.tex}
\subfile{bh-9-1-9.tex}
\subfile{bh-9-6-1.tex}
\subfile{bh-9-7.tex}



\section{Assignment 6}
\subfile{bh-9-25.tex}
\subfile{bh-9-37.tex}
\subfile{bh-9-50.tex}
\subfile{bh-10-2-3.tex}


%\section{Assignment 7}
\subfile{bh-10-9.tex}
\subfile{bh-10-28.tex}
\subfile{bh-10-30.tex}
\subfile{bh-10-39.tex}

\clearpage
\section{Challenges}
These problems are for fun, but not needed for the exam. You won't get extra points for it. However, from past years we know that the good students really liked these challenges.
\subfile{poisson.tex}
\subfile{cauchy.tex}
\subfile{records.tex}
\subfile{normal.tex}
\subfile{recourse.tex}
\subfile{beluga.tex}
\subfile{benford.tex}
\subfile{betting.tex}




% \opt{justhints}{
% \Closesolutionfile{hint}
% \clearpage
% \section{Hints}
% \input{hint}
% \input{ans}
% }

\end{document}
