\documentclass[poll_tutorial_format]{subfiles}
\begin{document}
	\maketitle
		\setcounter{section}{1}
	\section{PT Week 2 ((Non-naive) probability and conditional probability)}
	
	\subsection{Set things up}
	\label{sec:set-things-up}
	
	
	
	\setcounter{theorem}{-1}

	\begin{exercise}
		Have you helped your neighbors to set up their polleverywhere app? 
		\begin{enumerate}
			\item Yes
			\item No
		\end{enumerate}
	\end{exercise}
	
	\subsection{Real questions}
	\label{sec:start-real-questions pt week 2}
			\begin{exercise}
		Suppose we have a sample space $S$ with subsets/events $A$ and $B$ such that $P(A)\neq 0$.
		Which one of these answers could be incorrect:
		\begin{enumerate}
			\item $P(B|A)=\frac{P(A\cap B)}{P(A)}$
			\item $P(A\cap B)=P(B|A)P(A)$
			\item $P(\emptyset)=0$
			\item $P(B|A)+P(B|A^c)=1$
			\item $P(B|A)+P(B^c|A)=1$
			\item $P(A|S)=P(A)$
			\item $P(B)=P(B|S)$
		\end{enumerate}
	\end{exercise}

	


\begin{exercise}
	Which one of these answers is not part of (or implied by) the definition of the non-naive probability?
	Choose one of these answers: 
	\begin{enumerate}
		\item $P(S)=1$
		\item $P(\cup_i A_i) = \sum_i P(A_i)$ for arbitrary events $A_i$'s
		\item $P(A\cup B)=P(A)+P(B)-P(A\cap B)$
		\item $P(\emptyset)=1-P(S)$
	\end{enumerate}
\end{exercise}



\begin{exercise}
	(Coin tossing problem) A fair coin is flipped two times, event $A$ represents both tosses landing heads and $B$ represents the event that the first toss landed tails. 
	Which of these answers is incorrect: 
	\begin{enumerate}
		\item $P(B\cup B^c )=P(B)+P(B^c)$
		\item $P(B\cup B^c |A)=P(B|A)+P(B^c|A)$
		\item $P(A|B^c)=P(B^c|A)$
		\item $P(A|B)=0$
		\item $P(A|B^c)=P(B)$
	\end{enumerate}
\end{exercise}


\begin{exercise}
	Here are some statements regarding the conditional probability (where $P(A)\neq 0$, and $S$ is the sample space).
	Which of these answers may be incorrect: 
	\begin{enumerate}
		\item $P(S|A)=1$
		\item $P(\cup_{i=1}^\infty B_i|A) = \sum_{i=1}^\infty P(B_i|A)$ for disjoint events $B_i$'s
		\item $P(B|A)+P(B^c|A)=1$
		\item $P(B|A)+P(B|A^c)=1$ 
		\item Suppose that $A$ and $B$ are independent then $P(B|A)=P(B)$ and $P(B|A)P(A)=P(B)P(A)$ 
	\end{enumerate}
\end{exercise}


\begin{exercise}
	Which of these answers may be incorrect: 
	\begin{enumerate}
		\item If $P(A|B)=P(A)$, then Event $A$ is independent from Event $B$.
		\item If Event $A$ is independent from Event $B$, then $P(A|B)=P(A)$ and $P(B|A)=P(B)$.
		\item If Event $A$ is independent from Event $B$, then $P(A\cap B)=P(A)P(B)$.
	\end{enumerate}
\end{exercise}


\begin{exercise}
	(Coin tossing problem) A fair coin is flipped two times, event $A$ represents both tosses landing heads, $B$ represents the event that the first toss landed tails and $C$ represents the event that the second toss landed tails. 
Which of these answers is incorrect: 
\begin{enumerate}
		\item $P(A)=P(A|B)P(B)+P(A|B^c)P(B^c)$ 
		\item $P(A|B^c)=1/2$
		\item $P(B^c\cap C^c)=P(B^c)P(C^c)$ 
		\item $P(B^c\cap C^c|A)=P(B^c|A)P(C^c|A) $  
	\end{enumerate}
\end{exercise}

\begin{exercise}
	(Coin tossing problem) A fair coin is flipped two times, event $A$ represents both tosses landing heads, $B$ represents the event that the first toss landed tails and $C$ represents the event that the second toss landed tails. 
	Which of these answers is correct: 
	\begin{enumerate}
		\item $P(A|B)=P(A)$  
		\item $P(B|B^c)=1$
		\item $P(B| C)=P(B)P(C)$  
		\item $P(B\cap C|A)=P(B|A)P(C|A)$ and hence $B$ and $C$ are conditionally independent given event $A$.  
	\end{enumerate}
\end{exercise}


\begin{exercise} 
Is the following statement right or wrong?\\~\\ Event $A$ is independent from Event $B$ if and only if $$P(A\cap B)=P(A)P(B),$$ and it is possible that $P(A|B)$ (or $P(B|A)$ or both) is not defined. 
	\begin{enumerate}
		\item Right.
		\item  Wrong.
		\item Neither of above.
	\end{enumerate}
\end{exercise}


\begin{exercise} 
	Which of these statements is correct: 
	\begin{enumerate}
		\item Independence of events $A$ and $B$ implies that $A$ and $B$ are conditional independent as well.
		\item If $A$ and $B$ are conditional independent given event implies that $A$ and $B$ are independent. 
		\item If $A$ and $B$ are independent, we always have $P(A|B)=P(A)$. 
		\item  $\emptyset$ is independent from any events $A$, though $P(A|\emptyset)$ is not defined.
	\end{enumerate}
\end{exercise}


\begin{exercise}
	Which of these statements is incorrect: 
	\begin{enumerate}
		\item If events $A$, $B$ and $C$ are independent, then we must have $P(A\cap B\cap C)=P(A)P(B)P(C)$.
		\item In practice, we use $P(A|B,C)$ to denote $P(A|B\cup C)$. 
		\item $P(T_3|T_1,T_2)=\frac{P(T_3\cap T_2|T_1)}{P(T_2|T_1)}$ where we assume $P(T_1\cap T_2)>0$. 
		\item $P(T_3|T_1,T_2)=\frac{P(T_3\cap T_1|T_2)}{P(T_1|T_2)}$ where we assume $P(T_1\cap T_2)>0$. 
	\end{enumerate}
\end{exercise}

	
 	
	
	
\end{document}
