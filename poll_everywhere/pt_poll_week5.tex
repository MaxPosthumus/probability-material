\documentclass[poll_tutorial_format]{subfiles}
\begin{document}
	\maketitle
	
	\setcounter{section}{4}
	\section{PT Week 5 (Continuous) random variables}
	
	\subsection{Set things up}
	\label{sec:set-things-up}
	
	
	
	\setcounter{theorem}{-1}
	\begin{exercise}
		Have you helped your neighbors to set up their polleverywhere app? 
		\begin{enumerate}
			\item Yes
			\item No
		\end{enumerate}
	\end{exercise}
	
	\subsection{Real questions}
	\label{sec:start-real-questions pt week 5}
	
		
	
 
	
	
	\begin{exercise}
		Which of these statements is incorrect: 
		\begin{enumerate}
			\item A r.v. has a continuous distribution if its CDF is differentiable except at finitely many points.   
			\item For a continuous r.v. $X$ with the given differentiable CDF
			$F$, the conventional probability density function (PDF) choice of $X$ is the derivative $f$ of the CDF.
			\item It is impossible for two different PDFs to have the same CDF.
			\item The distribution of a random variable is about the probability values of all events associated with the r.v..
		\end{enumerate}
	\end{exercise}
	
	
		\begin{exercise}
		Let $X$ be a continuous real-valued r.v. with probability continuous r.v. with PDF $f$ and $f(0)> 0$. Let $\tilde{f}$ be another probability density function such that  $\tilde{f}(x)=f(x)$ for all $x$ except $x=0$ where $\tilde{f}(0)=0$.   Suppose $\tilde{X}$ has the PDF $\tilde{f}$.	
		Which of these statements is incorrect: 
		\begin{enumerate}
			\item The CDF of $X$ is given by $F(x)=\int_{-\infty}^{x}f(x)dx$.
			\item The support of $X$ contains the number $0$, but the support of $\tilde{X}$ does not.
			\item $X$ and $\tilde{X}$ have the same distribution, e.g., $\P{X \leq a }=\P{\tilde{X} \leq a}$ for any real number $a$.
 			\item $X$ and $\tilde{X}$ have different distributions.
 			\item $\P{X=0}=\P{\tilde{X}=0}=0$.
		\end{enumerate}
	\end{exercise}
	
	
	\begin{exercise}
		Which of these statements is incorrect: 
		\begin{enumerate}
			\item If two r.v.s have the same CDF (or PDF or PMF), then they have the same distribution.
			\item If two r.v.s have the same distribution, then they have the same CDF/PMF (but not necessarily the same PDF) .
			\item CDF functions map to values within 0 and 1, while PDF functions map to non-negative values. 
			\item If $f$ is a PDF function for a real-valued r.v.,  then $\int_{-\infty}^{x}f(s)ds$ is a CDF function and  $\int_{-\infty}^{+\infty}f(s)ds=1$.
			\item CDF functions are continuous functions. 
		\end{enumerate}
	\end{exercise}

\begin{exercise}
	Which of these statements may be incorrect: 
	\begin{enumerate}
		\item Given a CDF $F$ for a continuous r.v., which is differentiable except at points $a_1, a_2, \cdots$, we conventionally (unless otherwise specified) choose its PDF to be $f(x)=F'(x)$ except for  $f(a_1), f(a_2), \cdots$ that are set to be 0.
		\item The discrete r.v.'s PMF is uniquely determined by one CDF but there might be several (could be infinite) PDFs that correspond to the same CDF.   
		\item It is possible to have two r.v.s with the same CDF (or in other words, same distribution) but different PDFs and supports. 
		\item If two r.v.s have the same distribution, they must be equal to each other.
	\end{enumerate}
\end{exercise}
	
	

	\begin{exercise}
		Let $X$ be a continuous real-valued r.v. with probability continuous r.v. with PDF $f$. Let $F(x)=\int_{-\infty}^x f(s)ds$. 
		Choose the correct statement:
		\begin{enumerate}
			\item $\P{a< X\leq b} =F(b)-F(a)$.
			\item $\P{a< X\leq b} =\int_{a}^b f(s)ds$.
			\item $\P{a< X\leq b}=\P{a\leq  X\leq b}=\P{a< X< b}$.
			\item $\int_{-\infty}^{+\infty} f(x)dx=1$.
			\item All of the above. 
		\end{enumerate}
	\end{exercise}
	
	
	
	\begin{exercise}
	Let $X$ be a continuous real-valued r.v. with probability continuous r.v. with PDF $f$. Let $F(x)=\int_{-\infty}^x f(s)ds$. Let $\tilde{f}$ be another probability density function such that  $\tilde{f}(x)=f(x)$ for all $x$ except $x=0$ where $\tilde{f}(0)\neq {f}(0)$.
	Choose the correct statement:
	\begin{enumerate}
		\item $\P{a< X\leq b|X<a} =0$.
		\item $\P{a< X\leq b|X\leq c} =\frac{\int_{a}^b f(s)ds}{\int_{-\infty}^c f(s)ds}$ (suppose $\P{X\leq c} >0, c>b$).
		\item $F(x)=\int_{-\infty}^x \tilde{f}(s)ds$.
		\item $\E{X}= \int_{-\infty}^{+\infty} xf(x)dx=1$.
		\item It is possible for $f(x)$ to be larger than 10.
		\item All of the above. 
	\end{enumerate}
\end{exercise}

	
	
	\begin{exercise}
	Let $U\sim $Unif(0,1), and $X=F^{-1}(U)$ with $F$ being the CDF of Unif(0,1) such that 
	$$F(x)= \begin{cases}0 & \text { if } x \leq 0 \\ x & \text { if } 0<x<1 \\ 1 & \text { if } x \geq 1\end{cases} $$
	Which of these statements is incorrect:
	\begin{enumerate}
		\item $\E{U}=\int_{0}^{1}xdx=1/2$.
		\item $\V{U}=\int_{0}^{1}(x-1/2)^2dx$.
		\item $\V{aU}=a^2\E{U-\E{U}}^2$.
		\item $X$, $U$ and $F(U)$ have different distributions. 
	\end{enumerate}
\end{exercise}

	
	\begin{exercise}
	 	Let $U\sim $Unif(0,1), and denote $\tilde{U}=aU+a$, what is the value $\P{\tilde{U}\in(a,7a/4)|\tilde{U}\leq 3a/2}$?  
		Which of these statements is correct: 
		\begin{enumerate}
			\item 0
			\item 3/4
			\item 1
			\item 3/2
			\item -1/2 
		\end{enumerate}
	\end{exercise}
	
	
	\begin{exercise}
		Let $Z\sim$N(0,1) (standard normal distribution) and $X=\mu+\sigma Z$. 
		Which of these statements is incorrect: 
		\begin{enumerate}
			\item $X\sim N(\mu, \sigma^2)$
			\item The PDF of $X$ is $f(x)=\frac{1}{\sqrt{2\pi \sigma^2}} e^{-\frac{(x-\mu)^2 }{2\sigma^2 }}$
			\item $X$ and $Z$ have the same support: the whole real line.   
			\item The PDF function of $X$ is symmetric about 0.
		\end{enumerate}
	\end{exercise}
	

	\begin{exercise}
		Let $Z\sim$N(0,1) with CDF $\Phi$ and pdf $\psi$ (standard normal distribution) and $X=\mu+\sigma Z$. 
		Which of these statements may be incorrect: 
		\begin{enumerate}
			\item $\P{X\leq \mu)}=\P{X\geq \mu}=1/2$.
			\item The CDF of $X$ is $\Phi((x-\mu)/\sigma)$.
			\item The PDF of $X$ is $\phi((x-\mu)/\sigma)/\sigma$.
			\item $\P{|X-\mu|\leq 2\sigma} >\P{|Z|<2}$.
			\item $\P{X-\mu=\sigma Z}=1$.
		\end{enumerate}
	\end{exercise}
	
 
	
	
	
	
\end{document}
